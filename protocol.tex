\todo[inline]{The protocol is abstract...}

A node has a pseudoidentifier, which is assumed to be unique. Connections are made based on this.

Nodes provide some method of creating and handing requests for: which active nodes they know about; what packets they are in possession of, and the class of these packets; and what proof of work algorithms and cryptographic methods are accepted (figure \ref{fig:protointernodalconusecase}).

Nodes are provided with a list of initial nodes' psueudoidentifiers. Connections are make to these nodes and queried to discover the rest of the network and what methods are accepted.

Each node is classified into one of 4 categories: active, where the node is online and responding the requests normally; inactive, where the node appears to be offline; unknown, where the node is yet to be tested for activity; undiscovered, where the node has not been discovered or referenced.

All newly discovered nodes are considered unknown and tested to reclassify them as either active or inactive. Active nodes can then be used for connections, creating multiple unidirectional links to the network. Should an active node appear to be inactive by not responding appropriately, it shall be reclassified as inactive. Inactive nodes are periodically checked for signs of activity and reclassified if necessary. \todo[inline]{Flow for classification?}

Information about packet possession is achieved through selective polling. Each node holds some mapping from nodes to timestamps, which relates to the last time that the mapped node was polled. This timestamp is sent with any polling request, therefore ensuring the list of packets the polled node lists is composed purely of new information. \todo[inline]{Polling seq diagram}

\todo{Move to elements?}While to process of inserting a packet into the network could easily be achieved through just possessing the packet and letting other nodes retrieve it, this both increases the period of distribution (decreasing immediate availability), and as the packet will originate from a single node, may allow the origin of a packet to be determined.

3 packet classes are used: meta packets, for exchanging protocol information between nodes; data packets, for the representation of larger messages; and broadcast packets, to represent small messages, or a notification of the existence and location of a data packet.

%\begin{figure}[!h]
%	\begin{center}
%		\begin{tikzpicture} % [show background grid ]
%		
%		\umlclass[simple=false,type=class,x=0,y=1]{StorAPI}{stor : StorNode}{send(identity : PublicIdentity, message : binary)\\receive(identity : PrivateIdentity) : binary}
%		\umlclass[simple=false,type=abstract,x=0,y=-7]{Identity}{}{}
%		\umlclass[simple=false,type=abstract,x=4,y=-4]{PublicIdentity}{}{\umlvirt{encrypt(data : binary) : binary}}
%		\umlclass[simple=false,type=abstract,x=-4,y=-4]{PrivateIdentity}{}{\umlvirt{decrypt(data : binary) : binary}}
%		
%		\umluniaggreg[attr1=1|sends for, attr2=0..*|,pos1=0.2,pos2=1.0]{StorAPI}{PublicIdentity}
%		
%		\umluniaggreg[attr1=1|listens for, attr2=0..*|,pos1=0.2,pos2=0.8]{StorAPI}{PrivateIdentity}
%		
%		\umlcompo[attr1=1|, attr2=1|]{Identity}{PublicIdentity}
%		
%		\umlcompo[attr1=1|, attr2=1|]{Identity}{PrivateIdentity}
%		\end{tikzpicture}
%	\end{center}
%	\caption{API Class Diagram\label{fig:api-uml-design}}
%\end{figure}

Distribution of packets is mainly achieved through polling and requesting packets that are not yet possessed, however, packet forwarding is also permitted. All broadcast packets should be acquired for checking, but the same is not true for data packets. Any valid forwarded data packet should be retained, but it is not nessesary, although allowable, to also acquire data packets through requests. In this respect, the proportion of data packets that node desires can be varied.

To achieve packet insertion, the node forwards the packet on to its connected nodes. The inserting node may or may not retain and advertise possession of the forwarded packet.

Broadcast and data packets should be accompanied with appropriate proof-of-works. Processing of a packet should only occur if the proof-of-work is valid for the packet. The validity of a proof-of-work is left for the implementation details.

Additional packets containing junk data may also be inserted into the network, 