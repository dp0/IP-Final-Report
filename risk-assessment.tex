\begin{landscape}
	\subsection{Risk Analysis and Contingency Plan}
	\paragraph{Workflow}
	\begin{center}
		\newcolumntype{R}[1]{>{\raggedright\let\newline\\\arraybackslash\hspace{0pt}}m{#1}}
		\small{
		\begin{tabular}{| R{3.5cm} | l | l | l | R{9cm} |}
			\hline
			\textbf{Risk} & \textbf{Impact (1-5)} & \textbf{Probability (1-5)} & \textbf{Exposure (1-5)} & \textbf{Mitigation} \\ \hline
			Data loss & 4 & 1 & 4 & Make regular backups and store in separate physical locations to offset damage from fires/floods etc. \\ \hline
			Bug in work & 2 & 3 & 6 & Use source control with regular commits to ensure when something goes wrong, it can be easily rectified. \\ \hline
			Development environment failure & 2 & 2 & 4 & Save changes on a regular basis, reducing the amount of damage that can occur from an IDE crash or similar situation. \\ \hline
			Network failure & 3 & 2 & 6 & Use another suitable network connection. As a last resort, a mobile data connection could be used. \\ \hline
			Temporary computer failure & 3 & 1 & 3 & Use a lab machine or borrow a laptop from the university. \\ \hline
			Interfering work & 3 & 2 & 6 & Ensure all deadlines are known in advance and manage time accordingly so all work can be completed to an acceptable standard in plenty of time. \\ \hline
			Underestimation of work completion time & 3 & 3 & 9 & Be conservative when planning time for unfamiliar work. Plan periods of time that allow for catching up. \\ \hline
			Absent supervisor/examiner & 3 & 1 & 3 & Approach tutor for advice. \\ \hline
			\end{tabular}}
	\end{center}
	
	\newpage
	
	\paragraph{Project Content}
	\begin{center}
		\newcolumntype{R}[1]{>{\raggedright\let\newline\\\arraybackslash\hspace{0pt}}m{#1}}
		\small{
		\begin{tabular}{| R{3.5cm} | l | l | l | R{9cm} |}
			\hline
			\textbf{Risk} & \textbf{Impact (1-5)} & \textbf{Probability (1-5)} & \textbf{Exposure (1-5)} & \textbf{Mitigation} \\ \hline
			Design changes & 2 & 4 & 8 & Ensure design is as modular as possible, making it easy to change any element. \\ \hline
			Project difficulties & 3 & 2 & 6 & Speak to supervisor/tutor about issues \\ \hline
			Disagreements with supervisor & 2 & 1 & 2 & Speak to supervisor about issues \\ \hline
			Impossible concept & 5 & 1 & 5 & Have redundancy plans in the event that some concepts turn out to be impossible \\ \hline
			tor made unavailable & 4 & 1 & 4 & Plan system around a different anonymity network \\ \hline
			\end{tabular}}
	\end{center}
	
	\paragraph{Other}
	\begin{center}
		\newcolumntype{R}[1]{>{\raggedright\let\newline\\\arraybackslash\hspace{0pt}}m{#1}}
		\small{
		\begin{tabular}{| R{3.5cm} | l | l | l | R{9cm} |}
			\hline
			\textbf{Risk} & \textbf{Impact (1-5)} & \textbf{Probability (1-5)} & \textbf{Exposure (1-5)} & \textbf{Mitigation} \\ \hline
			Family emergency & 3 & 1 & 3 & Stop work for a period of and continue after the emergency. Plan periods of catch-up time to negate any affects. \\ \hline
			Health problems & 4 & 1 & 4 & If the issue is not too serious, continue work if possible. Otherwise use catch-up time to negate affects. \\ \hline
			Health implications of working with computers & 1 & 2 & 2 & Follow UK HSE guidance on working with computers and VDUs. \\ \hline
			\end{tabular}}
	\end{center}
		
\end{landscape}