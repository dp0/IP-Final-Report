% Diagrams produced from modified code originally made by andrino Claudio 2011 - http://claudiofiandrino.altervista.org/

\definecolor{pinegreen}{cmyk}{0.92,0,0.59,0.25}
\definecolor{royalblue}{cmyk}{1,0.50,0,0}
\definecolor{lavander}{cmyk}{0,0.48,0,0}
\definecolor{violet}{cmyk}{0.79,0.88,0,0}

\tikzstyle{legend_overlay}=[rectangle, rounded corners, thin,
top color= white,bottom color=white, draw=black,
minimum width=2.5cm, minimum height=0.8cm,
black,rotate=90]
\tikzstyle{legend_general}=[rectangle, rounded corners, thin,
top color=white,bottom color=white, draw=black,
minimum width=2.5cm, minimum height=0.8cm,
black]
\begin{figure}
	\centering
	\begin{tikzpicture}[auto, thick]
	\tikzstyle{vertex}=[circle,fill=black!25,minimum size=17pt,inner sep=0pt]

	\node[legend_overlay] at (-6.5,0){\textsc{Before Split}};
	\node[legend_overlay] at (-6.5,-5){\textsc{After Split}};
	
	\node[cloud, draw=black, cloud puffs=16, cloud puff arc= 100, minimum width=7cm, minimum height=4.5cm, aspect=1] (Stream-main) at (0,0) {};
	\node[legend_general] at (0,1.6){Stream for PKs 0* to f*};
	\foreach \name/\y/\pk in {a/0/PK: 0fa79\ldots, b/0.7/PK: c2955\ldots, c/1.4/PK: 3795a\ldots, d/2.1/PK: ee775\ldots} {
		\node[vertex] (A-\name) at (-1.3,0.8-\y) {$\name$};	\node[right] (A-label-a) at (-1,0.8-\y) {$\pk$};
	}
	
	\node[cloud, draw=black, cloud puffs=16, cloud puff arc= 100, minimum width=5cm, minimum height=3.5cm, aspect=1] (Stream-split1) at (-3,-5) {};
	\node[legend_general] at (-3,-3.9){{Stream for PKs 0* to 7*}};
	\foreach \name/\y/\pk in {a/0/PK: 0fa79\ldots, c/0.7/PK: 3795a\ldots} {
		\node[vertex] (B-\name) at (-4.3,-5-\y) {$\name$};	\node[right] (B-label-a) at (-4,-5-\y) {$\pk$};
	}
	
	\node[cloud, draw=black, cloud puffs=16, cloud puff arc= 100, minimum width=5cm, minimum height=3.5cm, aspect=1] (Stream-split2) at (3,-5) {};
	\node[legend_general] at (3,-3.9){{Stream for PKs 8* to f*}};
	\foreach \name/\y/\pk in {b/0/PK: c2955\ldots, d/0.7/PK: ee775\ldots} {
		\node[vertex] (C-\name) at (1.7,-5-\y) {$\name$};	\node[right] (C-label-a) at (2.0,-5-\y) {$\pk$};
	}
	
	\draw[-latex, thick, black] (Stream-main) -- (Stream-split1);
	\draw[-latex, thick, black] (Stream-main) -- (Stream-split2);
	\end{tikzpicture}
	\caption{Example of intended stream usage}
	\label{fig:stream-normal}
%	\end{center}
\end{figure}


\begin{figure}
	\centering
	\begin{tikzpicture}[auto, thick]
	\tikzstyle{vertex}=[circle,fill=black!25,minimum size=17pt,inner sep=0pt]
	
	\node[cloud, fill=white, draw=black, cloud puffs=16, cloud puff arc= 100, minimum width=7cm, minimum height=4.5cm, aspect=1] at (0,0) {};
	\node[legend_general] at (0,1.6){{Stream for PKs 0fa*}};
	
	\foreach \name/\y/\pk in {E/0/PK: 0fa73\ldots, F/0.7/PK: 0fa18\ldots, H/2.1/PK: 0fa9c\ldots} {
		\node[vertex, fill=black!70!white, text=white] (A-\name) at (-1.3,0.8-\y) {$\name$};
		\node[right, text=black] (A-label-a) at (-1,0.77-\y) {$\pk$};
	}
	
	\foreach \name/\y/\pk in {g/1.4/PK: 0faf2\ldots} {
		\node[vertex] (A-\name) at (-1.3,0.8-\y) {$\name$};
		\node[right] (A-label-a) at (-1,0.8-\y) {$\pk$};
	}
	\end{tikzpicture}
	\caption{Example of malicious nodes isolating a legitimate node}
	\label{fig:stream-malicious}
	%\end{center}
\end{figure}


%\begin{figure}
%	\centering
%	\begin{tikzpicture}[auto, thick]
%	\tikzstyle{vertex}=[circle,fill=black!25,minimum size=17pt,inner sep=0pt]
%		\node[vertex] (Node-1) at (-2,0) {$1$};
%		\node[vertex] (Node-2) at (2,1.5) {$2$};
%		\node[vertex] (Node-3) at (2,0) {$3$};
%		\node[vertex] (Node-4) at (2,-1.5) {$4$};
%		
%		\node[vertex] (Node-X) at (5,2) {$X$};
%		\node[right] (Node-label-X) at (5.5,2) {Node X};
%		
%	\tikzstyle{vertex}=[circle,fill=yellow!25,draw=black,minimum size=12pt,inner sep=0pt]
%		
%		\node[vertex] (Packet-y) at (5,1.2) {$y$};
%		\node[right] (Packet-label-y) at (5.5,1.2) {Packet y};
%	
%		\node[vertex] (Packet-1-a) at (-2.5,0.5) {$a$};
%		\node[vertex] (Packet-1-b) at (-2.7,0) {$b$};
%		%\node[vertex] (Packet-1-c) at (-2.5,-0.5) {$c$};
%		\draw[thick, black] (Node-1) -- (Packet-1-a);
%		\draw[thick, black] (Node-1) -- (Packet-1-b);
%		%\draw[thick, black] (Node-1) -- (Packet-1-c);
%		
%		\node[vertex] (Packet-2-a) at (2.5,2) {$a$};
%		\node[vertex] (Packet-2-b) at (2.7,1.5) {$b$};
%		\draw[thick, black] (Node-2) -- (Packet-2-a);
%		\draw[thick, black] (Node-2) -- (Packet-2-b);
%		
%		\node[vertex] (Packet-3-a) at (2.5,0.5) {$a$};
%		\node[vertex] (Packet-3-c) at (2.7,0) {$c$};
%		\draw[thick, black] (Node-3) -- (Packet-3-a);
%		\draw[thick, black] (Node-3) -- (Packet-3-c);
%		
%		\node[vertex] (Packet-4-b) at (2.5,-1) {$b$};
%		\node[vertex] (Packet-4-d) at (2.7,-1.5) {$d$};
%		\draw[thick, black] (Node-4) -- (Packet-4-b);
%		\draw[thick, black] (Node-4) -- (Packet-4-d);
%		
%		\node[text width=5cm] at (6,-1) {Node 1 selects nodes 2 and 3 to be its connections into the network};
%		
%		\draw[-latex, thick, dashed, black] (4.5,0.4) -- (5.4,0.4);
%		\node[right] (Arrow) at (5.5,0.4) {Connection};
%		
%		\draw[-latex, thick, dashed, black] (Node-1) -- (Node-2);
%		\draw[-latex, thick, dashed, black] (Node-1) -- (Node-3);
%		%\node[right] (A-label-a) at (-1,0.8-\y) {$\pk$};
%	\end{tikzpicture}	
%	
%	\vspace*{0.1cm} \hrulefill \vspace*{0.1cm}
%	
%		\begin{tikzpicture}[auto, thick]
%		\tikzstyle{vertex}=[circle,fill=black!25,minimum size=17pt,inner sep=0pt]
%		\node[vertex] (Node-1) at (-2,0) {$1$};
%		\node[vertex] (Node-2) at (2,1.5) {$2$};
%		\node[vertex] (Node-3) at (2,0) {$3$};
%		\node[vertex] (Node-4) at (2,-1.5) {$4$};
%		
%		\tikzstyle{vertex}=[circle,fill=yellow!25,draw=black,minimum size=12pt,inner sep=0pt]
%		
%		\node[vertex] (Packet-1-a) at (-2.5,0.5) {$a$};
%		\node[vertex] (Packet-1-b) at (-2.7,0) {$b$};
%		\node[vertex] (Packet-1-N) at (-2.5,-0.5) {$N$};
%		\draw[thick, black] (Node-1) -- (Packet-1-a);
%		\draw[thick, black] (Node-1) -- (Packet-1-b);
%		\draw[thick, black] (Node-1) -- (Packet-1-N);
%		
%		\node[vertex] (Packet-2-a) at (2.5,2) {$a$};
%		\node[vertex] (Packet-2-b) at (2.7,1.5) {$b$};
%		\draw[thick, black] (Node-2) -- (Packet-2-a);
%		\draw[thick, black] (Node-2) -- (Packet-2-b);
%		
%		\node[vertex] (Packet-3-a) at (2.5,0.5) {$a$};
%		\node[vertex] (Packet-3-c) at (2.7,0) {$c$};
%		\draw[thick, black] (Node-3) -- (Packet-3-a);
%		\draw[thick, black] (Node-3) -- (Packet-3-c);
%		
%		\node[vertex] (Packet-4-b) at (2.5,-1) {$b$};
%		\node[vertex] (Packet-4-d) at (2.7,-1.5) {$d$};
%		\draw[thick, black] (Node-4) -- (Packet-4-b);
%		\draw[thick, black] (Node-4) -- (Packet-4-d);
%		
%		\node[text width=5cm] at (6,0) {Node 1 inserts a new packet N into the network by forwarding it along the connections, using Nodes 2 and 3 as rendezvous points};
%		
%		\draw[thick, red] (Node-1) -- (Node-2);
%		\draw[thick, red] (Node-1) -- (Node-3);
%		
%		\draw[-latex, thick, dashed, black] (Node-1) -- (Node-2);
%		\draw[-latex, thick, dashed, black] (Node-1) -- (Node-3);
%		%\node[right] (A-label-a) at (-1,0.8-\y) {$\pk$};
%		\end{tikzpicture}
%	
%	
%	
%	\caption{TEST}
%	\label{fig:TEST}
%	%\end{center}
%\end{figure}














