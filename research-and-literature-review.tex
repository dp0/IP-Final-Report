\section{Research and Literature Review}
	\subsection{Existing Systems and Technologies}
		\subsubsection*{Bitmessage}\label{bitmessage_existing}
			Bitmessage is a P2P communications protocol that enables parties to communicate with messages. Each party\footnote{A party may or not be a participating node.} holds at most 8 connections to other nodes in the network. Inserting a message into the network is done by encrypting the message for the recipient and forwarding it along all the connections. Nodes store messages for 2 days before they are removed. To retrieve a message, each message must be acquired and decrypted. Although Bitmessage's aim is a secure communications network, it does not address anonymity, having the undesirable affect that it is possible to tell who is using the network. Although it is possible to run Bitmessage over Tor, a node doing this is not able to accept connections~\cite{bitmsg}.
		\subsubsection*{Freenet}\label{freenet_existing}
			Freenet is an anonymous storage network that attempts to implement its own anonymity layer. Files are inserted and retrieved by relaying requests through successive nodes until a request is successful or until the request's TTL\footnote{Time to live, a value that is decremented every hop.} expires. As a file is successfully relayed back to the requester, each node also caches it~\cite{clarke2001freenet}.
			
			Because Freenet uses direct connections between nodes, the limiting factor for network speed is the slowest node in the request chain. However, if all nodes are reasonably fast, Freenet has a high potential for fast transfer speeds.
		\subsubsection*{Tor}
			Tor is an anonymity network that implements onion routing to obscure traffic patterns. The Tor network consists of relays run by volunteers. While most Tor relays only route traffic around the Tor network, some (exit relays) allow networks to be made to the wider Internet from within the Tor. Under typical usage conditions, onion routing will select a collection of these relays from the public list of relays, using an exit relay as the last relay if the destination is on the wider Internet. The data stream is then encrypted with successive layers of asymmetric encryption using the relay public keys so that when a relay receives some data, it can decrypt it and forward it to the next destination~\cite{dingledine2004tor}.
			
			A key feature of Tor is its hidden services, which are services that are hard to physically locate, meaning that it is possible to run services anonymously. Tor's hidden services provide identity verification using the 16 character onion address and as all communications stay within Tor, all traffic is encrypted. The technical specification of hidden service creation, discovery and rendezvous is described by~\cite{tor_rend}.
			%\todo[inline]{Tor}
			
		%\subsubsection*{BitTorrent}
		%	\todo[inline]{BitTorrent}
		\subsubsection*{Proof-of-Work}
			Achieving consensus on a traditional network may be possible where there is a link between user and some available identity. This can be achieved on an anonymous network where a trusted central authority is introduced that assigns identities to users, although doing so also introduces a central point of failure. For a distributed anonymous network, consensus can also be reached through the use of proof-of-work algorithms.
			
			A proof-of-work is a guarantee that some resource has been used to create it. In many cases, this will in the form of a CPU-bound proof-of-work where a heavy computation is required to produce a proof that is easy to verify. Originally described in \cite{back2002hashcash}, Hashcash uses a CPU-bound proof-of-work to compute partial hash inversions. While this is common, CPU-bound proof-of-works can typically be computed quickly on Graphical Processing Units (GPUs), and even faster on Application Specific Integrated Circuits (ASICs). This has a high potential to cause a rift between those using low power devices and those able to create or afford hardware such as ASICs. This gap has been observed with the proof-of-work system used by Bitcoin~\cite{peck2013bitcoin}.
			
			Memory-bound proof-of-works are a potential solution to this issue, where memory access latency is the bottleneck for generation. As this latency difference between cheaper and more expensive device is less extreme than with processing difference for CPU-bound proof-of-works, this has significant potential to address inequality of availability for those generating proof-of-works, increasing accessibility to those that cannot afford expensive hardware. An implementation of such a system is described by Tromp ~\cite{cuckoo}.
	\subsection{Problems and Issues}
		\subsubsection*{Availability of Information}
			The lifetime of data is one metric of availability, it defines how long some data will persist on the network before becoming unavailable. The distribution of data is also an important metric of availability, which is measured by the number of active nodes that hold some information.
			
			Networks such as Freenet approach this by increasing the availability for popular data, having nodes as part of the request chain cache the data. In this scenario, the lifetime is strongly correlated to the distribution. Bitmessage on the other hand takes an approach where distribution is universal, but data expires after 2 days, purposely decreasing availability to free network resources.
		\subsubsection*{Visibility of the Network}
			Networks that do not address anonymity often allow the harvesting of nodes, as node identities (such as IP addresses) are publicly visible. For an anonymous system, this is unacceptable and some form of masking is required.
			
			If some communicating parties can all be trusted and known to each other, it is possible to create a small network based just on these parties. Freenet has support for such networks, calling them darknets. While suitable for small groups where everyone is trusted, if the requirement for the network is public availability, where parties cannot be trusted, then darknets cannot be used.
		\subsubsection*{Deanonymization Attacks}
			In the past, anonymous services deemed to be secure have been subject to deanonymization attacks. An example of this is the take down of Silk Road, a black market hosted using Tor's hidden services. Although there is a question over how Silk Road was taken down, given that it was a centralised service, with a single party in control, it could not be considered secure, as removing just 1 server removed the entire service's availability. 
		\subsubsection*{Trust}
			Nodes can be malicious, they can attempt to cheat the protocol, perhaps by reporting that they are participating and sharing a resource, but in reality they could be doing nothing. Networks relying on participation have to deal with this by classifying nodes are malicious or not. This could come in the form of a simple mechanism where nodes that misbehave $n$ times are blacklisted, or it could be a more complex machine learning algorithm.
		\subsubsection*{Sybil Attacks} \label{sybil_problems}
			On any network, it may be possible for one user to control many identities, and therefore pretending to be many users. This can cause issues in any system attempting to use consensus based on identity. The typical remedy for this attack is to link a high-cost resource to identities. IPv4 addresses are commonly used for this purpose, as there is a relatively small finite pool and spoofing is considered difficult. IPv6, with it's relatively large pool of addresses may not offer such protection~\cite{cholez2009evaluation}.
			
			Due to the nature of anonymous networks, it is not possible to use identifiers, such as IP addresses, as a resource. Instead, different attack negation mechanisms must be used. A trusted central authority could assign identities which could then be used on the anonymous network, but introduced a centralised point of failure. Proof-of-works may offer a solution, where elements of a system with any form of consensus can be protected with a guarantee that a valid proof-of-work required some resource to be used. A requirement for a proof-of-work for all communications could render a large Sybil attack difficult \cite{borisov2006computational}, although this could decrease network availability to some users.
		
			Ultimately, while some methods are able detect or prevent some Sybil attacks, Douceur reasons that "without a logically centralized
			authority, Sybil attacks are always possible" \cite{douceur2002sybil}.
		\subsubsection*{Spamming and Denial of Service Attacks}
			While partially related to Sybil attacks, denial of service attacks attempt to limit the availability of the network. In the context of a distributed storage network, this could quite easily come from a user constantly inserting data into the network and therefore exhausting the network's storage capacity.
			
			To prevent this kind of attack, some form of consensus is required to determine what data is legitimate. This therefore has the same issue of Sybil attacks.
	\subsection{Cryptography}
		Encryption come in two forms, symmetric and asymmetric. A symmetric cipher is one where the same key is used for both encryption and decryption. If used for communications between multiple parties, this requires that the key has been distributed between the parties, which can be problematic. Asymmetric ciphers on the other hand work on two keys, a public and private key. To encrypt a message, the public key is used, while decryption is only possible with the private key. \cite{appliedcrypto}
		
		Padding is a cryptographic mechanism that further adds to confidentiality. If for instance a sender were to use asymmetric encryption to send an encrypted message to two other parties using their respective public keys. If one of these parties were to decrypt the ciphertext with their private key and re-encrypt it with the other recipient's public key and match this to a previously intercepted communication, it would be possible to use this known-plaintext attack to deduce the communication that has occurred between the sender and the other party. This can be prevented through the use of a padding scheme, where the message is padded with some value and then encrypted. Upon decryption, the padding is then removed. Through this mechanism, it is no longer possible to determine the intended recipient of a ciphertext just from knowing the plaintext and an intercepted communication.
		
		Padding also offers protection against traffic analysis, where it may be possible to determine communication content from the traffic's behaviour. This is shown by Wright, where the lack of padding is shown to weaken an encrypted VoIP link \cite{Wright:2010:USP:1880022.1880029}.
	
		Confidentiality by encryption is described by RFC-4880 through the use of a hybrid cryptosystem \cite{network2007openpgp}. This is achieved through the generation of a random symmetric key which is used to encrypt the message, possibly along with a digital signature. This random key is then encrypted with the recipient's public key. Both the symmetrically encrypted message and the asymmetrically encrypted key can then be sent to the intended recipient.